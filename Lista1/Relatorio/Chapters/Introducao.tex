\section{Introdu\c{c}\~{a}o} \label{sec:intro}
A primeira lista da disciplina de M\'etodos Num\'ericos compreende uma revis\~ao de c\'alculo, abordando temas como, regra da cadeia e os  operadores gradiente, divergente e rotacional. A seguir, uma breve descri\c{c}\~ao do que foi discutido em aula.

A regra da cadeia \'e uma t\'ecnica utilizada para diferenciar fun\c{c}\~oes compostas por outras fun\c{c}\~oes. O operador gradiente, quando aplicado em uma fun\c{c}\~ao escalar, gera um vetor que aponta para a dire\c{c}\~ao de maior crescimento da fun\c{c}\~ao. O operador divergente representa a densidade volum\'etrica de fluxo que sai de um campo vetorial de um volume infinitesimal em torno de um ponto. Por fim, o operador roatcional pode ser definido como a densidade de circula\c{c}\~ao de um campo vetorial em torno de um ponto, representado pelo vetor cuja dire\c{c}\~ao e magnitude denotam o eixo e a magnitude da circula\c{c}\~ao m\'axima.

A lista foi dividida em quatro exerc\'icios, cada um abordando um dos temas supracitados. A seguir, ser\~ao apresentados os exerc\'icios e suas respectivas solu\c{c}\~oes. Real\c{c}a-se que, durante toda a realiza\c{c}\~ao dos exerc\'icios, usou-se as refer\^encias \cite{stewart2007essential} e \cite{becker1981finite} para consulta.