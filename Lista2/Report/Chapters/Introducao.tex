\section{Introduction} \label{sec:introduction}
The need of numerically integrate a function arises for evaluate the definite integral of a function that might have explicit antiderivative or whose antiderivative is not easy to obtain. In numerical methods, such as the Finite Element Method, the numerical integration is used to evaluate the integration that composes the weak formulation of the problem. 
 
The main ideia behind numerical integration is to approximate the integral of a function by the sum of weights times the function evaluated at some points (see Eq. \eqref{eq:general_integral})
\begin{equation}
    \int_{a}^{b} f(x) dx \approx \sum_{i=1}^{n} w_i f(x_i).
    \label{eq:general_integral}
\end{equation}
in which $w_i$ are the weights and $x_i$ are the points where the function is evaluated.

The examples herein developed approach several techiniques designed to numerically integrate a function. This list comprehends the Trapzoidal, Simpson's (1/3 and 3/8) and Gauss-Legendre rules. Equations \eqref{eq:trapzoidal} - \eqref{eq:gauss_legendre} present each method, respectively.
\begin{equation}
    \int_{a}^{b} f(x) dx \approx b-a \left[ \frac{1}{2}f(a) + \frac{1}{2}f(b) \right],
    \label{eq:trapzoidal}
\end{equation}
\begin{equation}
    \int_{a}^{b} f(x) dx \approx \frac{b-a}{2} \left[ \frac{1}{3}f(a) + \frac{4}{3}f\left(\frac{a+b}{2}\right) + \frac{1}{3}f(b) \right],
    \label{eq:simpson_1_3}
\end{equation}
\begin{equation}
    \int_{a}^{b} f(x) dx\approx (b-a) \left[ \frac{1}{8}f(a) + \frac{3}{8}f\left(\frac{2a+b}{3}\right) + \frac{3}{8}f\left(\frac{a+2b}{3}\right) + \frac{1}{8}f(b) \right],
    \label{eq:simpson_3_8}
\end{equation}
\begin{equation}
    \int_{a}^{b} f(x) dx \approx \sum_{i=1}^{n} \text{DetJac} f\left(x(\xi_i)\right) w_i,
    \label{eq:gauss_legendre}
\end{equation}
in which, for Eq. \eqref{eq:gauss_legendre}, the mapping function $x(\xi_i)$ is given by Eq. \eqref{eq:mapping_function}
\begin{equation}
    x(\xi_i) = \frac{1 - \xi_i}{2} a + \frac{1 + \xi_i}{2} b,
    \label{eq:mapping_function}
\end{equation}
and DetJac is the Jacobian determinant, given by Eq. \eqref{eq:jacobian_determinant}
\begin{equation}
    \text{DetJac} = \frac{b-a}{2},
    \label{eq:jacobian_determinant}
\end{equation}

The following refereces were used during the elaboration of this report: \cite{becker1981finite}, \cite{burden2005student}, and \cite{cunhametodos}.