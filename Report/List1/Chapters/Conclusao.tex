\section{Conclus\~ao} 
Para o primeiro m\'odulo de m\'etodos num\'ericos foi realizado uma breve revis\~ao de derivadas, regra da cadeia e dos operadores gradiente, divergente e rotacional. Esses temas s\~ao de suma import\^ancia por comporem, na maioria das vezes, as equa\c{c}\~oes diferenciais que se deseja resolver. 

Com os exerc\'icios 1 e 2, p\^ode-se provar e compreender propriedades fundamentais entre os operadores gradiente, divergente e rotacional. Os exerc\'icios 3 e 4 foram respons\'aveis por abordarem a regra da cadeia em fun\c{c}\~oes com mais de uma vari\'avel. Em especial, o exerc\'icio 4 se assemelha muito com fun\c{c}\~oes mapeamento, que s\~ao amplamente utilizadas em integra\c{c}\~ao num\'erica e problemas de elementos finitos, temas que ainda ser\~ao abordados no curso.

O exerc\'icio 5 treina uma vis\~ao mais pr\'atica dos conceitos de divergente e rotacional. Ao n\~ao passar as equa\c{c}\~oes dos campos vetoriais, o exerc\'icio se torna mais desafiador, j\'a que \'e esperada a capacidade de identificar o comportamento do campo vetorial apenas visualmente. Por fim, conclui-se que o m\'odulo foi de grande valia para a compreens\~ao dos conceitos fundamentais de c\'alculo vetorial e que ser\~ao de grande import\^ancia para o desenvolvimento do curso.