\section{Conclusion}\label{sec:conclusion}
The numerical integration of three different functions is presented in this work. The functions are numerically integrated using the Trapezoidal, the Simpson's 1/3 and 3/8, and the Gauss-Legendre rules. The results obtained by these methods point that, although the code herein developed is able to integrate the functions, the error convergence is not as expected. This is due to the fact that the functions chosen for this work are not well-behaved. 

During the discussion, it is considered the mean value for all cases, unless the opposite is stated. Function 1 depends on the parameter $\epsilon$, which the lesser its value, the more like a Dirac Delta function it behaves. For $\epsilon = 1$, the error convergence for all four method met the expected rate. However, for $\epsilon = 10^{-4}$, the results did not converge on the theortical rate. The same can be told about the second function, which contains the Sine Integral in its analytical solution. For the third function, the rate of convergence was close to the expected for all methods. 

Speaking of the rate of convergence inbetween intervals, only the first function with $\epsilon = 1$ presented a good convergence. Although the third equation showed good rates for almost every method. When it comes to the Gauss-Legendre rule with four points, the obtained rates were not as expected, although the error decreased quickly for the first and third functions. It is indiscutible that the Gauss-Legendre rule is the most accurate method. 

Another phenomenon observed was the behaviour of the error when its value gets closer to the machine precision, as mentioned before. From this moment on, there were cases when the error did not decrease, instead, it increased, which was expected. At this point, the approximated solution can be considered exact.

In summary, it is believed that the code herein developed is able to integrate functions. All the phenomena observed follow the expected from literature, and the results for well-behaved functions are in accordance with the theortical rate of convergence. The cases in which the error did not converge as expected are because the methods are not suitable for the functions chosen and therefore have limitations.