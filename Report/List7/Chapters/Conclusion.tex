\section{Conclusions}\label{sec:conclusions}
The present work aimed to solve the ordinary differential equation using the Runge-Kutta method. An algorithm is developed to read a given Butcher Tableau and approximate the solution for the ODE. The main difference between initial value problems and boundary value problems is discussed, and for the latter, a Galerkin algorithm is also implemented.

For the Galerkin method, a few comments are in order. Since the method is based on the integration of functions, the number of integration points when using the Gaussian Quadrature is key to the accuracy of the results. During the elaboration of the code, it was observed that, for a trigonometric body force and exact solution, the method suffered considerably from integrating accurately the functions. Only when 10 integration points were set the results were close to the exact solution. 

Another important aspect is that the domain herein used is coincident with the domain in which the Gaussian Quadrature is defined. Otherwise, the method would require the evaluation of the Jacobian to correctly integrate the functions changing the domain of integration. The Boundary Conditions imposed in the problem are also crucial, being the Dirichlet condition null on $\partial \Omega_D$ so that no extra BC methods are implemented. The Neumann BC is simply the inner product of the gradient of the state variable $u$ with the normal vector $\bm{n}$ to the edge of the domain. 

Finally, it is observed that the Least Squares Method indeed yields a more accurate solution than the linearization of the curve fitting the set of data points. Although the method for non-linear systems is more computationally expensive, its employment is justified when the linearization of the curve is not accurate enough.