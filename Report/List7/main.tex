\documentclass[a4paper, 12pt]{article}

%-----------------------------------------------------
%                      PACKAGES
% ----------------------------------------------------
\usepackage[T1]{fontenc}
\usepackage[latin9]{inputenc}
\usepackage{fancyhdr}
\usepackage{geometry}
\usepackage[colorlinks=true, linkcolor=blue,citecolor=blue, urlcolor=blue]{hyperref}
\usepackage{indentfirst}
\usepackage{graphicx}
\usepackage{float}
\usepackage{setspace}
\usepackage{amsmath}
\usepackage{amssymb}
\usepackage{multirow}
\usepackage[table,xcdraw]{xcolor}
\usepackage{colortbl}
\usepackage{tabularx,booktabs}
\usepackage{enumerate}
\usepackage{titlesec}
\usepackage{caption}
\usepackage{subcaption}
\usepackage{nicefrac, xfrac}
\usepackage{subfiles}
\usepackage{mathtools}
\usepackage{bm}
\usepackage{tikz}
\usepackage[mode=buildnew]{standalone}
\usepackage{systeme}
\usepackage{stmaryrd}
\usepackage{import}
\usepackage[english]{babel} 
\usepackage{times}
\usepackage{pgfplots}
\usepackage{makecell}
\usepackage{threeparttable}
\usepackage{ragged2e}
\usepackage{tocloft}
\usepackage{booktabs}
\usepackage[acronym]{glossaries}
\usepackage{pdfpages}
\usepackage{lipsum}
\usepackage{hyphenat} 
\usepackage{xspace}
\usepackage{abntex2cite}
\usepackage{pdflscape}
\usepackage{pythonhighlight}
\usepackage{listings}
\usepackage{makecell}
\usepackage{longtable}
\usepackage{adjustbox}
\usepackage{nicefrac}

%-----------------------------------------------------
%               DOCUMENT SETTINGS
% ----------------------------------------------------
\geometry{a4paper,top=30mm,bottom=20mm,left=30mm,right=20mm}

\pgfplotsset{compat=1.17} 

\numberwithin{equation}{section}
\counterwithin{figure}{section}
\counterwithin{table}{section}
% \numberwithin{lstlisting}{section}

% Defining spaces between lines
\setstretch{1.5} 

\graphicspath{{Figures/}}

\tikzset{
	font={\fontsize{11pt}{12}\selectfont}}

\setcounter{secnumdepth}{4}
\setcounter{tocdepth}{4}

% defining the height of table cells
\renewcommand{\arraystretch}{1.4}

% defining the gradient, divergence, and curl operators
\newcommand{\grad}[1]{\nabla #1}
\renewcommand{\div}[1]{\nabla \cdot #1}
\newcommand{\curl}[1]{\nabla \times #1}

\def\textcite#1{\citeauthoronline{#1} \cite{#1}}
%-----------------------------------------------------
%              PRE TEXTUAL ELEMENTS
% ----------------------------------------------------
\newenvironment{epigrafe}{\newpage\mbox{}\vfill\hfill\begin{minipage}[t]{0.5\textwidth}}
{\end{minipage}\newpage}

%-----------------------------------------------------
%              PYTHON CODE SETTINGS
% ----------------------------------------------------
\definecolor{codegreen}{rgb}{0,0.6,0}
\definecolor{codegray}{rgb}{0.5,0.5,0.5}
\definecolor{codepurple}{rgb}{0.58,0,0.82}
\definecolor{backcolour}{rgb}{0.95,0.95,0.92}

\lstdefinestyle{mystyle}{
    backgroundcolor=\color{backcolour},   
    commentstyle=\color{codegreen},
    keywordstyle=\color{magenta},
    numberstyle=\tiny\color{codegray},
    stringstyle=\color{codepurple},
    basicstyle=\ttfamily\footnotesize,
    breakatwhitespace=false,         
    breaklines=true,                 
    captionpos=b,                    
    keepspaces=true,                 
    numbers=left,                    
    numbersep=5pt,                  
    showspaces=false,                
    showstringspaces=false,
    showtabs=false,                  
    tabsize=2
}

\lstset{style=mystyle}
%-----------------------------------------------------
%                     DOCUMENT 
% ----------------------------------------------------
\begin{document}
% ------- WORK INFORMATION -------
\author{Carlos Henrique Chama Puga }
\newcommand{\RA}{195416}

\title{List 7 \\ IVB, BVP, Least Square Method}
\newcommand{\theme}{IVB, BVP, Least Square Method}

\newcommand{\Uni}{{Universidade Estadual de Campinas}}
\newcommand{\Fac}{{Faculdade de Engenharia Civil, Arquitetura e Urbanismo}}

\newcommand{\advisor}{Porf. Dr. Philippe Devloo}
\newcommand{\coadvisor}{Dr. Giovane Avancini}

\newcommand*{\workyear}{2024}

\makeatletter

% ------- PRE TEXTUAL PAGES -------
% Could not find another way to remove the page number from the first pages
\fancypagestyle{plain}{
    \fancyhf{}% Limpa todos os campos
    \fancyfoot[C]{}%
    \renewcommand{\headrulewidth}{0pt}%
}

\pagestyle{plain}
\def\logos{
    \noindent
    \raisebox{-.5\height}{\includegraphics[width=2.2cm]{Figures/logo-unicamp.pdf}}

    \vspace*{1.5cm}
    
    \noindent
    \begin{center} \large
        \MakeUppercase{\Uni}\\
        \Fac\\
    \end{center}
}

\def\openningpage{
  \logos
  \vskip 15mm
  \begin{center}
    \Large
    {IC639: M\'etodos Num\'ericos para Engenharia Civil}
    \vskip 20mm
      {\bf \@title}
  \end{center}
  \vskip 25mm
  \begin{flushright}
    \large
    {\textbf{Student:} \\ \@author - \RA}
    \vskip 10mm
    {{\bf{Advisors:}} \\
     \advisor \\
     \coadvisor}
  \end{flushright}
    \vfill
    \large
  \begin{center}
    Campinas\\\workyear
  \end{center}
}

\openningpage % Cover page

% Table of contents
\newpage
\tableofcontents 

% From here on, the page number is shown
% Chapter first page settings
\newpage
\fancypagestyle{plain}{
    \fancyhf{}% Limpa todos os campos
    \fancyhead[R]{\thepage}%
    \renewcommand{\headrulewidth}{0pt}%
}

\fancypagestyle{headings}{%
    \fancyhf{}% Limpa todos os campos
    \fancyhead[L]{\theme}% Nome do trabalho à esquerda
    \fancyhead[R]{\thepage}% Numero da página à direita
    \renewcommand{\headrulewidth}{1pt}%
}

% ------- CHAPTERS -------
\pagestyle{headings}
\section{Introduction} \label{sec:introduction}
It is possible to think of a matrix as a linear transformation applied to a vector. When the matrix is applied to a vector and the resulting vector is parallel to the original vector, this vector is said to be an eigenvector of the matrix so that 
\begin{equation}
    A\mathbf{v} = \lambda \mathbf{v},
    \label{eq:eigenvalue}
\end{equation}
in which $A$ is the matrix, $\mathbf{v}$ is the eigenvector, and $\lambda$ is the eigenvalue. 

Equation \eqref{eq:eigenvalue} can be rewritten as
\begin{equation}
    (A - \lambda I)\mathbf{v} = 0,
    \label{eq:eigenvalue_analytical}
\end{equation}
which means that, for any nonzero vector $\mathbf{v}$, Eq. \eqref{eq:eigenvalue_analytical} holds only if the determinant of the matrix $A - \lambda I$ is zero. In other words, matrix $A - \lambda I$ is singular. This is the characteristic equation of the matrix $A$ and can be used to find the eigenvalues of the matrix. 

Naturally, the eigenvalues of a matrix are the roots of the characteristic equation. However, a generalization of the characteristic equation is not always possible, and therefore, its analytical solution is not always feasible. 

The Power Method is an iterative method that can be used to find the most prominent eigenvalue of a matrix. The method is simple to implement and can be used to find both the eigenvalue and the eigenvector of the matrix. 

For a given vector $\mathbf{y}$ linearly independent of $\mathbf{v}$, it is possible to write
\begin{equation}
    \mathbf{y} = \sum_{i=1}^{n}\beta_i\mathbf{v}_i,
    \label{eq:linear_combination}
\end{equation}
in which $\mathbf{v}_i$ are the eigenvectors of the matrix $A$. 

Multiplying Eq. \eqref{eq:linear_combination} by $A^k$ yields
\begin{equation}
    A^k\mathbf{y} = \sum_{i=1}^{n}\beta_i\lambda_i^k\mathbf{v}_i,
    \label{eq:eigenvalue_power}
\end{equation}
and factoring Eq. \eqref{eq:eigenvalue_power} by $\lambda_1^k$ gives
\begin{equation}
    {A^k\mathbf{y}} = {\lambda_1^k}\sum_{i=1}^{n}\beta_i\left(\frac{\lambda_i}{\lambda_1}\right)^k\mathbf{v}_i.
    \label{eq:eigenvalue_power_factor}
\end{equation}

Since $\lambda_1$ is the most prominent eigenvalue, the term $\left(\frac{\lambda_i}{\lambda_1}\right)^k$ tends to zero for $i \neq 1$. Therefore, the term $\sum_{i=1}^{n}\beta_i\left(\frac{\lambda_i}{\lambda_1}\right)^k\mathbf{v}_i$ tends to $\beta_1\lambda_1^k\mathbf{v}_1$ as $k$ tends to infinity. 

This is the main idea behind the Power Method. The method starts with an initial guess $\mathbf{v}_0$. The matrix $A$ is multiplied by $\mathbf{v}_0$ to find the next eigenvector
\begin{equation}
    \mathbf{y} = A\mathbf{v}_{i-1},
    \label{eq:y}
\end{equation}
the norm of the result is calculated and considered the new eigenvalue
\begin{equation}
    \lambda_i = \left\|\mathbf{y}\right\|,
\end{equation}
the new eigenvector is the previous result divided by the new eigenvalue
\begin{equation}
    \mathbf{v}_i = \frac{\mathbf{y}}{\lambda_i},
\end{equation}

This process is repeated until the error between the current and previous eigenvalues is smaller than a given precision. Since Eq. \eqref{eq:y} is performed at each iteration, the limit when $k$ tends to infinity is approached.

In the next sections, the power method is implemented and the results are discussed. The following bibliography is referred to in this work: \cite{de2000metodos,burden1997numerical}.
\section{NonLinearSolver Implementation} \label{sec:nonlinear_solver_implementation}
\section{Results} \label{sec:results}
This section presents the main results achieved by the proposed methods. The results are divided into two parts: the first one presents the values obtained by the numerical integration using the methods implemented in this work, using 5 refinement levels and the rate of convergence. The second part compares the numerical and the analytical results, discussing whether or not the error convergence is as expected. 


\subsection{Numerical Integration Results}
\begin{table}[H]
    \centering
    \caption{Trapezoidal Rule: Numerical Integration.}
    \begin{tabular}{ccc|ccc}
    \hline
    \multicolumn{3}{c}{\textbf{Function 1  -} $\bm{\epsilon = 1}$} & \multicolumn{3}{c}{\textbf{Function 1 -} $\bm{\epsilon = 10^{-4}}$} \\ \hline
    N. Intervals & Approx. Sol. & $|| e ||$ & N. Intervals & Approx. Sol. & $|| e ||$ \\ \hline
    1 & 0.73575 & 0.75788 & 1 & 0.0 & 0.01772 \\
    2 & 1.36787 & 0.12576 & 2 & 1.0 & 0.98227 \\
    4 & 1.46274 & 0.03541 & 4 & 0.5 & 0.48227 \\
    8 & 1.48596 & 0.010195 & 8 & 0.25 & 0.23227 \\
    16 & 1.49173 & 0.00253 & 16 & 0.125 & 0.10727 \\
    32 & 1.49316 & 0.00063 & 32 & 0.0625 & 0.04477 \\ \hline
    \multicolumn{3}{c}{\textbf{Function 2}} & \multicolumn{3}{c}{\bf{Function 3}} \\ \hline
    N. Intervals & Approx. Sol. & $|| e ||$ & N. Intervals & Approx. Sol. & $|| e ||$ \\ \hline
    1 &  {-0.00267} &  {0.00177} & 1 &  {9.42822} &  {2.21083} \\
    2 &  {-0.00287} &  {0.00197} & 2 &  {11.56799} &  {1.5334} \\
    4 &  {-0.00129} &  {0.00243} & 4 &  {11.62885} &  {0.34341} \\
    8 &  {-0.00154} &  {0.00173} & 8 &  {11.63645} &  {0.09581} \\
    16 &  {-0.00087} &  {0.00064} & 16 &  {11.63840} &  {0.02383} \\
    32 &  {-0.00092} &  {0.00039} & 32 &  {11.63889} &  {0.00595} \\ \hline
    \end{tabular}
\end{table}
\begin{table}[H]
    \centering
    \caption{Trapezoidal Rule: Rate of Convergence.}
    \begin{tabular}{ccccccc}
        \hline
        \multirow{2}{*}{\textbf{Function}} & \multicolumn{5}{c}{\textbf{Subintervals}} & \multicolumn{1}{c}{\multirow{2}{*}{\textbf{Average Rate}}} \\ \cline{2-6}
 & (1 - 2) & (2 - 4) & (4 - 8) & (8 - 16) & (16 - 32) & \multicolumn{1}{c}{} \\ \hline
        1: $\epsilon = 1$ & 5.79 & -1.03 & -1.05 & -1.11 & -1.26 & 0.27 \\
        1: $\epsilon = 10^{-4}$ & 0.15 & 0.31 & -0.49 & -1.42 & -0.72 & -0.43 \\
        2 & -0.53 & -2.16 & -1.84 & -2.01 & -2.00 & -1.71 \\ 
        3 & 0.27 & -0.43 & -1.71 \\ \hline
    \end{tabular}
\end{table}
\begin{table}[H]
    \centering
    \caption{Simpson 1/3 Rule.}
    \begin{tabular}{ccc|ccc}
    \hline
    \multicolumn{3}{c}{\textbf{Function 1  -} $\bm{\epsilon = 1}$} & \multicolumn{3}{c}{\textbf{Function 1 -} $\bm{\epsilon = 10^{-4}}$} \\ \hline
    N. Intervals & Approx. Sol. & $|| e ||$ & N. Intervals & Approx. Sol. & $|| e ||$ \\ \hline
    1 & 1.57858 & 0.08493 & 1 & 1.33333& 1.31560 \\
    2 & 1.49436 & 0.00071 & 2 & 0.33333 & 0.31560 \\
    4 & 1.49371 & 0.00029 & 4 & 0.16666 & 0.14894 \\
    8 & 1.49365 & 1.74e-05 & 8 & 0.08333 & 0.06560 \\
    16 & 1.49364 & 1.07e-06 & 16 & 0.04166 & 0.02394 \\
    32 & 1.49364 & 6.70e-08 & 32 & 0.02083 & 0.00311 \\ \hline
    \multicolumn{3}{c}{\textbf{Function 2}} & \multicolumn{3}{c}{\bf{Function 3}} \\ \hline
    N. Intervals & Approx. Sol. & $|| e ||$ & N. Intervals & Approx. Sol. & $|| e ||$ \\ \hline
    1 & -0.00293  & 0.00203  & 1 & 12.28124 &  0.64219 \\
    2 & -0.00077  & 0.00210  & 2 & 11.64914 &  0.05324 \\
    4 & -0.00162  & 0.00113  & 4 & 11.63898  & 0.00222 \\
    8 & -0.00064 & 0.00059  & 8 &  11.63905 &  0.00015 \\
    16 & -0.00094 & 0.00021 & 16 & 11.63905 &  9.52e-06 \\
    32 & -0.00082 & 8.50e-05  & 32 & 11.63905  & 5.93e-07  \\ \hline
    \end{tabular}
\end{table}
\begin{table}[H]
    \centering
    \caption{Simpson 3/8 Rule.}
    \begin{tabular}{ccc|ccc}
    \hline
    \multicolumn{3}{c}{\textbf{Function 1  -} $\bm{\epsilon = 1}$} & \multicolumn{3}{c}{\textbf{Function 1 -} $\bm{\epsilon = 10^{-4}}$} \\ \hline
    N. Intervals & Approx. Sol. & $|| e ||$ & N. Intervals & Approx. Sol. & $|| e ||$ \\ \hline
    1 & 1.52619 & 0.03255 & 1 & 0.0 & 0.01772 \\
    2 & 1.49398 & 0.00033 & 2 & 0.25 & 0.23227 \\
    4 & 1.49367 & 0.00013 & 4 & 0.125 & 0.10727 \\
    8 & 1.49365 & 7.75e-06 & 8 & 0.0625 & 0.04477 \\
    16 & 1.49364 & 4.7e-07 & 16 & 0.03125 & 0.01352 \\
    32 & 1.49364 & 2.97e-08 & 32 & 0.01623 & 0.00148 \\ \hline
    \multicolumn{3}{c}{\textbf{Function 2}} & \multicolumn{3}{c}{\bf{Function 3}} \\ \hline
    N. Intervals & Approx. Sol. & $|| e ||$ & N. Intervals & Approx. Sol. & $|| e ||$ \\ \hline
    1 & 0.00148  & 0.00238 & 1 & 11.91167 & 0.27261  \\
    2 & -0.00035  & 0.00075 & 2 & 11.63937 & 0.01908  \\
    4 & -0.00065 & 0.00028 & 4 & 11.63902 & 0.00098 \\
    8 & -0.00077 & 0.00015 & 8 &  11.63905 &  6.82e-05 \\
    16 & -0.00082 & 0.00013 & 16 & 11.63905 &  4.23e-06 \\
    32 & -0.00089 & 3.67e-05  & 32 & 11.63905 & 2.63e-07  \\ \hline
    \end{tabular}
\end{table}

\begin{table}[H]
    \centering
    \caption{Simpson Gaussian Quadrature - 1 point.}
    \begin{tabular}{ccc|ccc}
    \hline
    \multicolumn{3}{c}{\textbf{Function 1  -} $\bm{\epsilon = 1}$} & \multicolumn{3}{c}{\textbf{Function 1 -} $\bm{\epsilon = 10^{-4}}$} \\ \hline
    N. Intervals & Approx. Sol. & $|| e ||$ & N. Intervals & Approx. Sol. & $|| e ||$ \\ \hline
    1 & 2.0 & 0.50635 & 1 & 2.0 & 1.98227 \\
    2 & 1.5576 & 0.06395 & 2 & 0.0 & 0.01772 \\
    4 & 1.50919 & 0.01815 & 4 & 3.68e-272 & 0.01772 \\
    8 & 1.49749 & 0.00511 & 8 & 6.92e-69 & 0.01772 \\
    16 & 1.49460 & 0.00127 &  16 & 2.71e-18 & 0.01772 \\
    32 & 1.49388 & 0.00031 & 32 & 7.17e-06 & 0.01771 \\ \hline
    \multicolumn{3}{c}{\textbf{Function 2}} & \multicolumn{3}{c}{\bf{Function 3}} \\ \hline
    N. Intervals & Approx. Sol. & $|| e ||$ & N. Intervals & Approx. Sol. & $|| e ||$ \\ \hline
    1 & -0.00306 & 0.00216 & 1 & 13.70776 & 2.06870\\
    2 & 0.00027& 0.00315 & 2 & 11.68971 & 0.84657\\
    4 & -0.00178 & 0.00194 & 4 & 11.64404 & 0.17504\\
    8 & -0.00020 & 0.00163 & 8 & 11.64035 & 0.04813\\
    16 & -0.00098 & 0.00055 & 16 & 11.63938 & 0.01193\\
    32 & -0.00077 & 0.00029 & 32 & 11.63914& 0.00297 \\\hline
    \end{tabular}
\end{table}

\begin{table}[H]
    \centering
    \caption{Simpson Gaussian Quadrature - 2 points.}
    \begin{tabular}{ccc|ccc}
    \hline
    \multicolumn{3}{c}{\textbf{Function 1  -} $\bm{\epsilon = 1}$} & \multicolumn{3}{c}{\textbf{Function 1 -} $\bm{\epsilon = 10^{-4}}$} \\ \hline
    N. Intervals & Approx. Sol. & $|| e ||$ & N. Intervals & Approx. Sol. & $|| e ||$ \\ \hline
    1 & 1.43306 & 0.06058 & 1 & 0.0 & 0.01772 \\ 
    2 & 1.49318 & 0.00045 & 2 & 1.12e-194 & 0.01772 \\
    4 & 1.49360 & 0.00019 & 4 & 1.62e-49 & 0.01772 \\
    8 & 1.49364 & 1.16e-05 & 8 & 1.88e-13 & 0.01772 \\
    16 & 1.49364 & 7.17e-07 & 16 & 0.00011 & 0.01760 \\
    32 & 1.49364 & 4.46e-08 & 32 & 0.01092 & 0.00680  \\ \hline
    \multicolumn{3}{c}{\textbf{Function 2}} & \multicolumn{3}{c}{\bf{Function 3}} \\ \hline
    N. Intervals & Approx. Sol. & $|| e ||$ & N. Intervals & Approx. Sol. & $|| e ||$ \\ \hline
    1 & -0.00066 & 0.00023 & 1 & 11.14786 & 0.49119 \\
    2 & -0.00135 & 0.00099 & 2 & 11.63349 & 0.03471 \\
    4 & -0.00069 & 0.00150 & 4 & 11.63910 & 0.00148 \\
    8 & -0.00112 & 0.00023 & 8 & 11.63906 & 0.00010 \\
    16 & -0.00081 & 0.00015 & 16 & 11.63905 & 6.34e-06 \\
    32 & -0.00095 & 7.48e-05 & 32 & 11.63905 & 3.95e-07 \\ \hline
    \end{tabular}
\end{table}

\begin{table}[H]
    \centering
    \caption{Simpson Gaussian Quadrature - 3 points.}
    \begin{tabular}{ccc|ccc}
    \hline
    \multicolumn{3}{c}{\textbf{Function 1  -} $\bm{\epsilon = 1}$} & \multicolumn{3}{c}{\textbf{Function 1 -} $\bm{\epsilon = 10^{-4}}$} \\ \hline
    N. Intervals & Approx. Sol. & $|| e ||$ & N. Intervals & Approx. Sol. & $|| e ||$ \\ \hline
    1 & 1.49867 & 0.00503 & 1 & 0.88888 & 0.87116 \\
    2 & 1.49362 & 1.90e-05 & 2 & 3.82e-56 & 0.01772 \\
    4 & 1.49364 & 1.16e-06 & 4 & 4.49e-15 & 0.01772 \\
    8 & 1.49364 & 1.66e-08 & 8 & 4.95e-05 & 0.01767 \\
    16 & 1.49364 & 2.55e-10 & 16 & 0.00954 & 0.00818  \\
    32 & 1.49364 & 4.05e-12 & 32 & 0.02114 & 0.00341 \\ \hline
    \multicolumn{3}{c}{\textbf{Function 2}} & \multicolumn{3}{c}{\bf{Function 3}} \\ \hline
    N. Intervals & Approx. Sol. & $|| e ||$ & N. Intervals & Approx. Sol. & $|| e ||$ \\ \hline
    1 & -0.00388 & 0.00298 & 1 & 11.66141&  0.02235 \\
    2 & -0.00104 & 0.00015 & 2 & 11.64364 & 0.00499 \\
    4 & -0.00120 & 0.00051 & 4 & 11.63905 & 5.18e-06 \\ 
    8 & -0.00079 & 0.00012 & 8 & 11.63905 & 8.91e-08 \\
    16 & -0.00099 & 0.00010 & 16 & 11.63905 & 1.37e-09 \\ 
    32 & -0.00087 & 4.45e-05 & 32 & 11.63905 & 2.14e-11 \\ \hline
    \end{tabular}
\end{table}

\begin{table}[H]
    \centering
    \caption{Simpson Gaussian Quadrature - 4 points.}
    \begin{tabular}{ccc|ccc}
    \hline
    \multicolumn{3}{c}{\textbf{Function 1  -} $\bm{\epsilon = 1}$} & \multicolumn{3}{c}{\textbf{Function 1 -} $\bm{\epsilon = 10^{-4}}$} \\ \hline
    N. Intervals & Approx. Sol. & $|| e ||$ & N. Intervals & Approx. Sol. & $|| e ||$ \\ \hline
    1 & 1.49366 & 1.56e-05 & 1 & 0.0 & 0.01772 \\
    2 & 1.49364 & 1.20e-08 & 2 & 1.0 & 0.98227 \\
    4 & 1.49364 & 1.28e-11 & 4 & 0.5 & 0.48227 \\
    8 & 1.49364 & 1.24e-14 & 8 & 0.25 & 0.23227 \\
    16 & 1.49364 & 4.44e-16 & 16 & 0.125 & 0.10727  \\
    32 & 1.49364 & 9.50e-16 & 32 & 0.0625 & 0.04477 \\ \hline
    \multicolumn{3}{c}{\textbf{Function 2}} & \multicolumn{3}{c}{\bf{Function 3}} \\ \hline
    N. Intervals & Approx. Sol. & $|| e ||$ & N. Intervals & Approx. Sol. & $|| e ||$ \\ \hline
    1 & -0.00267 & 0.00177 & 1 & 9.42822 & 2.21083 \\ 
    2 & -0.00287 & 0.00197 & 2 & 11.56799 & 1.5334  \\
    4 & -0.00129 & 0.00243 & 4 & 11.62885 & 0.34341 \\
    8 & -0.00154 & 0.00173 & 8 & 11.63645 & 0.09581  \\
    16 & -0.00087 & 0.00064 & 16 & 11.63840 & 0.02383 \\
    32 & -0.00092 & 0.00039 & 32 & 11.63889 & 0.00595 \\ \hline
    \end{tabular}
\end{table}

\begin{table}[H]
    \centering
    \caption{Trapezoidal Rule.}
    \begin{tabular}{ccccc}
    \hline
    \multicolumn{5}{c}{\textbf{Function 1}} \\
    \hline
    \multicolumn{3}{c}{\textbf{Numerical Integration}} & \multicolumn{2}{l}{\textbf{Rate of Convergence}} \\\hline
    \# Subintervals & Approx. Solution & $|| sol - sol_h ||$ & Subintervals & Rate \\\hline
    1 & 0.0 & 0.01772 & 1 - 2 & 5.79 \\
    2 & 1.0 & 0.98227 & 2 - 4 & -1.03 \\
    4 & 0.5 & 0.48227 & 4 - 8 & -1.05 \\
    8 & 0.25 & 0.23227 & 8 - 16 & -1.11 \\
    16 & 0.125 & 0.10727 & 16 - 32 & -1.26 \\
    32 & 0.0625 & 0.04477 & \textbf{Average Rate} & 0.27\\ \hline
    \multicolumn{5}{c}{\textbf{Function 2}} \\
    \hline
    \multicolumn{3}{c}{\textbf{Numerical Integration}} & \multicolumn{2}{l}{\textbf{Rate of Convergence}} \\\hline
    \# Subintervals & Approx. Solution & $|| sol - sol_h ||$ & Subintervals & Rate \\\hline
    1 & -0.00267 & 0.00177 & 1 - 2 & 0.15 \\
    2 & -0.00287 & 0.00197 & 2 - 4 & 0.31 \\
    4 & -0.00129 & 0.00243 & 4 - 8 & -0.49 \\
    8 & -0.00154 & 0.00173 & 8 - 16 & -1.42 \\
    16 & -0.00087 & 0.00064 & 16 - 32 & -0.72 \\
    32 & -0.00092 & 0.00039 & \textbf{Average Rate} & -0.43\\ \hline
    \multicolumn{5}{c}{\textbf{Function 3}} \\
    \hline
    \multicolumn{3}{c}{\textbf{Numerical Integration}} & \multicolumn{2}{l}{\textbf{Rate of Convergence}} \\\hline
    \# Subintervals & Approx. Solution & $|| sol - sol_h ||$ & Subintervals & Rate \\\hline
    1 & 9.42822 & 2.21083 & 1 - 2 & -0.53 \\
    2 & 11.56799 & 1.5334 & 2 - 4 &  -2.16\\
    4 & 11.62885 & 0.34341 & 4 - 8 &  -1.84\\
    8 & 11.63645 & 0.09581 & 8 - 16 &  -2.01\\
    16 & 11.63840 & 0.02383 & 16 - 32 &  -2.00\\
    32 & 11.63889 & 0.00595 & \textbf{Average Rate} & -1.71\\ \hline
\end{tabular}
\end{table}

\begin{table}[H]
    \centering
    \caption{Simpson 1/3 Rule.}
    \begin{tabular}{ccccc}
    \hline
    \multicolumn{5}{c}{\textbf{Function 1}} \\
    \hline
    \multicolumn{3}{c}{\textbf{Numerical Integration}} & \multicolumn{2}{l}{\textbf{Rate of Convergence}} \\\hline
    \# Subintervals & Approx. Solution & $|| sol - sol_h ||$ & Subintervals & Rate \\\hline
    1 & 1.33333 & 1.31560 & 1 - 2 &  -2.06\\
    2 & 0.33333 & 0.31560 & 2 - 4 &  -1.08\\
    4 & 0.16666 & 0.14894 & 4 - 8 &  -1.18\\
    8 & 0.08333 & 0.06560 & 8 - 16 &  -1.45\\
    16 & 0.04166 & 0.02394 & 16 - 32 &  -2.94\\
    32 & 0.02083 & 0.00311 & \textbf{Average Rate} & -1.74\\ \hline
    \multicolumn{5}{c}{\textbf{Function 2}} \\
    \hline
    \multicolumn{3}{c}{\textbf{Numerical Integration}} & \multicolumn{2}{l}{\textbf{Rate of Convergence}} \\\hline
    \# Subintervals & Approx. Solution & $|| sol - sol_h ||$ & Subintervals & Rate \\\hline
    1 & -0.00293 & 0.00203 & 1 - 2 &  0.05\\
    2 & -0.00077 & 0.00210 & 2 - 4 &  -0.89\\
    4 & -0.00162 & 0.00113 & 4 - 8 &  -0.92\\
    8 & -0.00064 & 0.00059 & 8 - 16 &  -1.50\\
    16 & -0.00094 & 0.00021 & 16 - 32 &  -1.32\\
    32 & -0.00082 & 8.50e-05 & \textbf{Average Rate} & -0.92\\ \hline
    \multicolumn{5}{c}{\textbf{Function 3}} \\
    \hline
    \multicolumn{3}{c}{\textbf{Numerical Integration}} & \multicolumn{2}{l}{\textbf{Rate of Convergence}} \\\hline
    \# Subintervals & Approx. Solution & $|| sol - sol_h ||$ & Subintervals & Rate \\\hline
    1 & 12.28124 & 0.64219 & 1 - 2 &  -3.59\\
    2 & 11.64914 & 0.05324 & 2 - 4 &  -4.58\\
    4 & 11.63898 & 0.00222 & 4 - 8 &  -3.86\\
    8 & 11.63905 & 0.00015 & 8 - 16 &  -4.01\\
    16 & 11.63905 & 9.52e-06 & 16 - 32 &  -4.00\\
    32 & 11.63905 & 5.93e-07 & \textbf{Average Rate} & -4.01\\ \hline
\end{tabular}
\end{table}

\begin{table}[H]
    \centering
    \caption{Simpson 3/8 Rule.}
    \begin{tabular}{ccccc}
    \hline
    \multicolumn{5}{c}{\textbf{Function 1}} \\
    \hline
    \multicolumn{3}{c}{\textbf{Numerical Integration}} & \multicolumn{2}{l}{\textbf{Rate of Convergence}} \\\hline
    \# Subintervals & Approx. Solution & $|| sol - sol_h ||$ & Subintervals & Rate \\\hline
    1 & 0.0 & 0.01772 & 1 - 2 &  3.71\\
    2 & 0.25 & 0.23227 & 2 - 4 &  -1.11\\
    4 & 0.125 & 0.10727 & 4 - 8 &  -1.26\\
    8 & 0.0625 & 0.04477 & 8 - 16 &  -1.73\\
    16 & 0.03125 & 0.01352 & 16 - 32 &  -3.18\\
    32 & 0.01623 & 0.00148 & \textbf{Average Rate} & -0.71\\ \hline
    \multicolumn{5}{c}{\textbf{Function 2}} \\
    \hline
    \multicolumn{3}{c}{\textbf{Numerical Integration}} & \multicolumn{2}{l}{\textbf{Rate of Convergence}} \\\hline
    \# Subintervals & Approx. Solution & $|| sol - sol_h ||$ & Subintervals & Rate \\\hline
    1 & 0.00148 & 0.00238 & 1 - 2 &  -1.67\\
    2 & -0.00035 & 0.00028 & 2 - 4 &  -1.40\\
    4 & -0.00065 & 0.00028 & 4 - 8 &  -0.91\\
    8 & -0.00077 & 0.00015 & 8 - 16 &  -0.13\\
    16 & -0.00082 & 0.00013 & 16 - 32 &  -1.92\\
    32 & -0.00089 & 3.67e-05 & \textbf{Average Rate} & -1.20\\ \hline
    \multicolumn{5}{c}{\textbf{Function 3}} \\
    \hline
    \multicolumn{3}{c}{\textbf{Numerical Integration}} & \multicolumn{2}{l}{\textbf{Rate of Convergence}} \\\hline
    \# Subintervals & Approx. Solution & $|| sol - sol_h ||$ & Subintervals & Rate \\\hline
    1 & 11.91167 & 0.27261 & 1 - 2 &  -3.84\\
    2 & 11.63937 & 0.01908 & 2 - 4 &  -4.28\\
    4 & 11.63902 & 0.00098 & 4 - 8 &  -3.85\\
    8 & 11.63905 & 6.82e-05 & 8 - 16 &  -4.01\\
    16 & 11.63905 & 4.23e-06 & 16 - 32 &  -4.00\\
    32 & 11.63905 & 2.63e-07 & \textbf{Average Rate} & -4.00\\ \hline
\end{tabular}
\end{table}

\begin{table}[H]
    \centering
    \caption{Gaussian Quadrature - 1 point.}
    \begin{tabular}{ccccc}
    \hline
    \multicolumn{5}{c}{\textbf{Function 1}} \\
    \hline
    \multicolumn{3}{c}{\textbf{Numerical Integration}} & \multicolumn{2}{l}{\textbf{Rate of Convergence}} \\\hline
    \# Subintervals & Approx. Solution & $|| sol - sol_h ||$ & Subintervals & Rate \\\hline
    1 & 2.0 & 1.98227 & 1 - 2 &  -6.81\\
    2 & 0.0 & 0.01772 & 2 - 4 &  0.00\\
    4 & 3.68e-272 & 0.01772 & 4 - 8 &  0.00\\
    8 & 6.92e-69 & 0.01772 & 8 - 16 &  0.00\\
    16 & 2.71e-18 & 0.01772 & 16 - 32 &  0.00\\
    32 & 7.17e-06 & 0.01771 & \textbf{Average Rate} & -1.36\\ \hline
    \multicolumn{5}{c}{\textbf{Function 2}} \\
    \hline
    \multicolumn{3}{c}{\textbf{Numerical Integration}} & \multicolumn{2}{l}{\textbf{Rate of Convergence}} \\\hline
    \# Subintervals & Approx. Solution & $|| sol - sol_h ||$ & Subintervals & Rate \\\hline
    1 & -0.00306 & 0.00216 & 1 - 2 &  0.54\\
    2 & 0.00027& 0.00315 & 2 - 4 &  -0.69\\
    4 & -0.00178 & 0.00194 & 4 - 8 &  -0.25\\
    8 & -0.00020 & 0.00163 & 8 - 16 &  -1.57\\
    16 & -0.00098 & 0.00055 & 16 - 32 &  -0.92\\
    32 & -0.00077 & 0.00029 & \textbf{Average Rate} & -0.58\\ \hline
    \multicolumn{5}{c}{\textbf{Function 3}} \\
    \hline
    \multicolumn{3}{c}{\textbf{Numerical Integration}} & \multicolumn{2}{l}{\textbf{Rate of Convergence}} \\\hline
    \# Subintervals & Approx. Solution & $|| sol - sol_h ||$ & Subintervals & Rate \\\hline
    1 & 13.70776 & 2.06870 & 1 - 2 &  -1.29\\
    2 & 11.68971& 0.84657 & 2 - 4 &  -2.27\\
    4 & 11.64404 & 0.17504 & 4 - 8 &  -1.86\\
    8 & 11.64035 & 0.04813 & 8 - 16 &  -2.01\\
    16 & 11.63938 & 0.01193 & 16 - 32 &  -2.00\\
    32 & 11.63914& 0.00297 & \textbf{Average Rate} & -1.89\\ \hline
\end{tabular}
\end{table}

\begin{table}[H]
    \centering
    \caption{Gaussian Quadrature - 2 points.}
    \begin{tabular}{ccccc}
    \hline
    \multicolumn{5}{c}{\textbf{Function 1}} \\
    \hline
    \multicolumn{3}{c}{\textbf{Numerical Integration}} & \multicolumn{2}{l}{\textbf{Rate of Convergence}} \\\hline
    \# Subintervals & Approx. Solution & $|| sol - sol_h ||$ & Subintervals & Rate \\\hline
    1 & 0.0 & 0.01772 & 1 - 2 &  0.00\\
    2 & 1.12e-194 & 0.01772 & 2 - 4 &  0.00\\
    4 & 1.62e-49 & 0.01772 & 4 - 8 &  0.00\\
    8 & 1.88e-13 & 0.01772 & 8 - 16 &  -0.01\\
    16 & 0.00011 & 0.01760 & 16 - 32 &  -1.37\\
    32 & 0.01092 & 0.00680 & \textbf{Average Rate} & -0.28\\ \hline
    \multicolumn{5}{c}{\textbf{Function 2}} \\
    \hline
    \multicolumn{3}{c}{\textbf{Numerical Integration}} & \multicolumn{2}{l}{\textbf{Rate of Convergence}} \\\hline
    \# Subintervals & Approx. Solution & $|| sol - sol_h ||$ & Subintervals & Rate \\\hline
    1 & -0.00066 & 0.00023 & 1 - 2 &  2.06\\
    2 & -0.00135 & 0.00099 & 2 - 4 &  0.60\\
    4 & -0.00069 & 0.00150 & 4 - 8 &  -2.65\\
    8 & -0.00112 & 0.00023 & 8 - 16 &  -0.58\\
    16 & -0.00081 & 0.00015 & 16 - 32 &  -1.10\\
    32 & -0.00095 & 7.48e-05 & \textbf{Average Rate} & -0.33\\ \hline
    \multicolumn{5}{c}{\textbf{Function 3}} \\
    \hline
    \multicolumn{3}{c}{\textbf{Numerical Integration}} & \multicolumn{2}{l}{\textbf{Rate of Convergence}} \\\hline
    \# Subintervals & Approx. Solution & $|| sol - sol_h ||$ & Subintervals & Rate \\\hline
    1 & 11.14786 & 0.49119 & 1 - 2 &  -3.82\\
    2 & 11.63349 & 0.03471 & 2 - 4 &  -4.54\\
    4 & 11.63910 & 0.00148 & 4 - 8 &  -3.86\\
    8 & 11.63906 & 0.00010 & 8 - 16 &  -4.01\\
    16 & 11.63905 & 6.34e-06 & 16 - 32 &  -4.00\\
    32 & 11.63905 & 3.95e-07 & \textbf{Average Rate} & -4.05\\ \hline
\end{tabular}
\end{table}

\begin{table}[H]
    \centering
    \caption{Gaussian Quadrature - 3 points.}
    \begin{tabular}{ccccc}
    \hline
    \multicolumn{5}{c}{\textbf{Function 1}} \\
    \hline
    \multicolumn{3}{c}{\textbf{Numerical Integration}} & \multicolumn{2}{l}{\textbf{Rate of Convergence}} \\\hline
    \# Subintervals & Approx. Solution & $|| sol - sol_h ||$ & Subintervals & Rate \\\hline
    1 & 0.88888 & 0.87116 & 1 - 2 &  -5.62\\
    2 & 3.82e-56 & 0.01772 & 2 - 4 &  0.00\\
    4 & 4.49e-15 & 0.01772 & 4 - 8 &  0.00\\
    8 & 4.95e-05 & 0.01767 & 8 - 16 &  -1.11\\
    16 & 0.00954 & 0.00818 & 16 - 32 &  -1.26\\
    32 & 0.02114 & 0.00341 & \textbf{Average Rate} & -1.60\\ \hline
    \multicolumn{5}{c}{\textbf{Function 2}} \\
    \hline
    \multicolumn{3}{c}{\textbf{Numerical Integration}} & \multicolumn{2}{l}{\textbf{Rate of Convergence}} \\\hline
    \# Subintervals & Approx. Solution & $|| sol - sol_h ||$ & Subintervals & Rate \\\hline
    1 & -0.00388 & 0.00298 & 1 - 2 &  -4.27\\
    2 & -0.00104 & 0.00015 & 2 - 4 &  1.73\\
    4 & -0.00120 & 0.00051 & 4 - 8 &  -2.00\\
    8 & -0.00079 & 0.00012 & 8 - 16 &  -0.30\\
    16 & -0.00099 & 0.00010 & 16 - 32 &  -1.23\\
    32 & -0.00087 & 4.45e-05 & \textbf{Average Rate} & -1.21\\ \hline
    \multicolumn{5}{c}{\textbf{Function 3}} \\
    \hline
    \multicolumn{3}{c}{\textbf{Numerical Integration}} & \multicolumn{2}{l}{\textbf{Rate of Convergence}} \\\hline
    \# Subintervals & Approx. Solution & $|| sol - sol_h ||$ & Subintervals & Rate \\\hline
    1 & 11.66141&  0.02235 & 1 - 2 &  -2.16\\
    2 & 11.64364 & 0.00499 & 2 - 4 &  -9.91\\
    4 & 11.63905 & 5.18e-06 & 4 - 8 &  -5.86\\
    8 & 11.63905 & 8.91e-08 & 8 - 16 &  -6.01\\
    16 & 11.63905 & 1.37e-09 & 16 - 32 &  -6.00\\
    32 & 11.63905 & 2.14e-11 & \textbf{Average Rate} & -5.99\\ \hline
\end{tabular}
\end{table}

\begin{table}[H]
    \centering
    \caption{Gaussian Quadrature - 4 points.}
    \begin{tabular}{ccccc}
    \hline
    \multicolumn{5}{c}{\textbf{Function 1}} \\
    \hline
    \multicolumn{3}{c}{\textbf{Numerical Integration}} & \multicolumn{2}{l}{\textbf{Rate of Convergence}} \\\hline
    \# Subintervals & Approx. Solution & $|| sol - sol_h ||$ & Subintervals & Rate \\\hline
    1 & 0.56888 & 0.55116 & 1 - 2 &  -4.96\\
    2 & 6.57e-11 & 0.01772 & 2 - 4 &  -0.04\\
    4 & 0.00048 & 0.01724 & 4 - 8 &  -2.65\\
    8 & 0.01497 & 0.00275 & 8 - 16 &  0.26\\
    16 & 0.02101 & 0.00328 & 16 - 32 &  -3.05\\
    32 & 0.01732 & 0.00039 & \textbf{Average Rate} & -2.09\\ \hline
    \multicolumn{5}{c}{\textbf{Function 2}} \\
    \hline
    \multicolumn{3}{c}{\textbf{Numerical Integration}} & \multicolumn{2}{l}{\textbf{Rate of Convergence}} \\\hline
    \# Subintervals & Approx. Solution & $|| sol - sol_h ||$ & Subintervals & Rate \\\hline
    1 & -0.00091 & 2.01e-05 & 1 - 2 &  5.11\\
    2 & -0.00159 & 0.00069 & 2 - 4 &  -2.74\\
    4 & -0.00096 & 0.00010 & 4 - 8 &  -1.24\\
    8 & -0.00093 & 4.39e-05 & 8 - 16 &  0.01\\
    16 & -0.00086 & 4.43e-05 & 16 - 32 &  -0.14\\
    32 & -0.00090 & 4.01e-05 & \textbf{Average Rate} & 0.20\\ \hline
    \multicolumn{5}{c}{\textbf{Function 3}} \\
    \hline
    \multicolumn{3}{c}{\textbf{Numerical Integration}} & \multicolumn{2}{l}{\textbf{Rate of Convergence}} \\\hline
    \# Subintervals & Approx. Solution & $|| sol - sol_h ||$ & Subintervals & Rate \\\hline
    1 & 11.63120 & 0.00785 & 1 - 2 &  -2.24\\
    2 & 11.64072 & 0.00166 & 2 - 4 &  -27.19\\
    4 & 11.63905 & 1.08e-11 & 4 - 8 &  -9.55\\
    8 & 11.63905 & 1.45e-14 & 8 - 16 &  0.22\\
    16 & 11.63905 & 1.69e-14 & 16 - 32 &  0.68\\
    32 & 11.63905 & 2.71e-14 & \textbf{Average Rate} & -7.61\\ \hline
\end{tabular}
\end{table}

\subsection{Comparison between Numerical and Analytical Results}

\section{Conclusions} \label{sec:conclusions}
In this list, the power method for finding eigenvalues and eigenvectors was implemented. The method was tested with a random initial guess and a fixed initial guess. The results showed that the method converges, for this specific matrix, in less than 50 iterations for almost every eigenvalue. The code implementation could recover the eigenvalues with a precision of $10^{-14}$ for the majority of the eigenvalues.

One advantage of the power method is that it is simple to implement and computes both eigenvalue and eigenvector at the same time. However, it is necessary to have a full set of eigenvalues, which is not always the case. 

The power method is a good starting point for finding the most prominent eigenvalue of a matrix. A generalization can be made so that the method is extended to find any eigenvalue of the matrix, not only the most prominent one. 

Finally, the matrix decomposition was implemented, recovering the matrix A from the matrices $\Lambda$ and Q.

% ------- BIBLIOGRAPHY -------
\addcontentsline{toc}{section}{References}
\bibliographystyle{abntex2-num}
\bibliography{References}

% ------- APPENDIX -------
\appendix
\section{Reposit\'orio \textit{GitHub}}\label{sec:github}
O c\'odigo fonte deste relat\'orio e dos programas utilizados para a resolu\c{c}\~ao dos problemas propostos est\~ao dispon\'iveis no reposit\'orio \textit{GitHub} do autor. O reposit\'orio pode ser acessado atrav\'es do link \href{https://github.com/CarlosPuga14/MetodosNumericos_2024S1}{CarlosPuga14/MetodosNumericos\_2024S1}.

\section{Gradiente de $f(x(\xi, \eta), y(\xi, \eta))$}\label{sec:gradienteF}
A seguir apresenta-se o gradiente de $f(x(\xi, \eta), y(\xi, \eta))$
\begin{equation*}
    \grad{f} = \left(\frac{df}{d\xi}, \frac{df}{d\eta} \right), \text{~em que:}
\end{equation*} 

\newpage
\fancypagestyle{headings}{%
    \fancyhf{}% Limpa todos os campos
    % \fancyhead[L]{\theme}% Nome do trabalho à esquerda
    % \fancyhead[R]{\thepage}% Numero da página à direita
    \renewcommand{\headrulewidth}{0pt}%
}
\pagestyle{headings}

\begin{landscape}   
    \begin{equation}
        \begin{split}
            \frac{df}{d\xi} =& ~9 e^{\sin\left(\left(5\xi + \cos(\xi^2) + \sin(1 +\eta)\right) \left(7\eta - \cos(\eta^2) + \sin(\xi^3) \right)^3\right)} \xi^2 \cos(\xi^3) \cos\left(\left(5\xi + \cos(\xi^2) + \sin(1+\eta) \right) \left(7\eta - \cos(\eta^2) + \sin(\xi^3) \right)^3\right)\\
            &\left(5\xi + \cos(\xi^2) + \sin(1 + \eta)\right)\left(7\eta - \cos(\xi^2) + \sin(\xi^3)\right)^2 + \left(5 - 2\xi\sin(\xi^2)\right)\\
            &\left(e^{5\xi + \cos(\xi^2) + \sin(1 + \eta)} + e^{\sin\left((5\xi + \cos(\xi^2) + \sin(1 + \eta))\right) \left(7\eta -\cos(\eta^2) +\sin(\xi^3) \right)^3}\cos\left(\left(5\xi +\cos(\xi^2) + \sin(1 + \eta)\right) \left(7\eta - \cos(\eta^2) + \sin(\xi^3) \right)^3 \right)\right.\\
            &\left. \left(7 \eta - \cos(\eta^2) + \sin(\xi^3)^3\right) \vphantom{e^{{5\xi + \cos(\xi^2) + \sin(1 + \eta)}}}\right)
        \end{split},
    \end{equation}
    \begin{equation}
        \begin{split}
            \frac{df}{d\xi} =& ~3e^{\sin\left(\left(5\xi + \cos(\xi^2) + \sin(1 + \eta) \right)\left(7 \eta - \cos(\eta^2) +\sin(\xi^3) \right)^3 \right)} \\
            & \cos\left(\left(5\xi + \cos(\xi^2) + \sin(1 + \eta) \right)\left(7\eta -\cos(\eta^2)+\sin(\xi^3)\right)^3 \right) \\
            & \left(7 + 2\eta \sin(\eta^2)\right)\left(5\xi +\cos(\xi^2) +\sin(1 +\eta)\right)\left(7\eta -\cos(\eta^2) + \sin(\xi^3) \right)^2 + cos(1 + \eta)\\
            & \left(e^{5\xi + \cos(\xi^2) + \sin(1 + \eta)} + e^{\sin\left(\left(5\xi + \cos(\xi^2) +\sin(1+\eta)\right)\left(7\eta -\cos(\eta^2)+\sin(\xi^3)\right)^3\right)}\cos\left(\left(5\xi + \cos(\xi^2) + \sin(1+\eta)\right) \left(7\eta - \cos(\eta^2) + \sin(\xi^3)\right)^3\right)\right.\\
            &\left. \left(7\eta - \cos(\eta^2) + \sin(\xi^3)\right)^3 \vphantom{e^{5\xi + \cos(\xi^2) + \sin(1 + \eta)}}\right)
        \end{split}
    \end{equation}

\end{landscape}
\end{document}